\documentclass[12pt]{article}
\usepackage{framelab}
\usepackage{fullpage}

\begin{document}
\begin{center}
Notes for Framelab
\end{center}

Recall the definition of the divergent beam X-ray transform: 

\begin{align*}
(\mathcal{D} u)(a,\theta) = \int_0^\infty u(a+t\theta)dt,\quad a\in\R^n,\theta\in S^{n-1}
\end{align*}

$\mathcal{D}u$ can be extended as a homogenous function to $\R^n\times \R^n$ via 

\begin{align*}
(\mathcal{P}u)(\xi,\eta) = \int_0^\infty u\left(\xi+t\frac{\eta-\xi}{\vert \eta-\xi\vert}\right) dt ,\quad \xi,\eta\in\R^n
\end{align*} In this format, $\xi$ is interpreted as a source position and $\eta$ as a detector position, so that $\frac{\eta-\xi}{\vert \eta-\xi\vert}\in S^{n-1}$ is a unit vector pointing from $\xi$ to $\eta$.  

To parameterize $\mathcal{P}u$ for the cone beam geometry, we use 

\begin{equation}
\begin{aligned}[c]
\xi_1 &= \xi_1(\theta,a,b,\zeta) = R_s\cos\theta \\
\xi_2 &= \xi_2(\theta,a,b,\zeta) = R_s\sin\theta \\
\xi_3 &= \xi_3(\theta,a,b,\zeta) = h\theta + \zeta
\end{aligned}\qquad \qquad
\begin{aligned}[c]
\eta_1 &= \eta_1(\theta,a,b,\zeta) = -R_d\cos\theta-a\sin\theta \\
\eta_2 &= \eta_2(\theta,a,b,\zeta) = -R_d\sin\theta+a\cos\theta \\
\eta_3 &= \eta_3(\theta,a,b,\zeta) = h\theta + b + \zeta 
\end{aligned} \label{param}
\end{equation} $R_s,R_d$ are the source-to-isocenter and isocenter-to-detector distances, respectively.  $(a,b)$ are the coordinates on the detector plane.  $\theta$ is the gantry rotation angle, and $\zeta$ is a vertical offset.  $h=P/2\pi$ where $P$ is the helix pitch.  We thus obtain the cone beam transform: 
\begin{align*}
(\P_cu)(\theta,a,b,\zeta) = \int_0^\infty u\left(\xi+t\frac{\eta-\xi}{\vert \eta-\xi\vert}\right)d\xi
\end{align*}where $\xi=\xi(\theta,a,b,\zeta),\eta=\eta(\theta,a,b,\zeta)$ are as described in \eqref{param}.  For code validation purposes, we would like to compute $\P_cu$ for some simple functions $u$ when $\xi\in \Gamma\subset\R^3$ is a smooth curve. 

\paragraph{Characteristic Function of the Unit Ball}
The first example we will use is the characteristic function of the unit ball, $u = \chi_{B}(x)$.  The ray transform of $\chi_B(x)$ is simply the length of intersection of the ray $\overrightarrow{\xi\eta}$ with the unit ball.  To compute this length, we use the fact that on $\partial B$ we have
\begin{align*}
\vert x\vert ^2 = 1
\end{align*}  The equation of a line from $\xi$ to $\eta$ is given by 
\begin{align*}
x(t) = \xi+t\frac{\eta-\xi}{\vert \eta-\xi\vert}
\end{align*}thus to find the intersection points, we find where $\vert x(t)\vert^2 = 1$: 
\begin{align*}
\vert x(t)\vert ^2 = \left\vert \xi+t\frac{\eta-\xi}{\vert \eta-\xi\vert}\right\vert ^2  &= \left(\xi+t\frac{\eta-\xi}{\vert \eta-\xi\vert}\right)\cdot \left(\xi+t\frac{\eta-\xi}{\vert \eta-\xi\vert}\right)\\
& = t^2 + 2t\frac{\xi\cdot(\eta-\xi)}{\vert \eta -\xi\vert} + \vert \xi\vert^2
\end{align*}Thus $\vert x(t)\vert^2 = 1$ if 
\begin{align*}
t^2 + 2t\frac{\xi\cdot(\eta-\xi)}{\vert \eta -\xi\vert} + \vert \xi\vert^2-1=0
\end{align*}

This is a quadratic equation in $t$, so we can find the entry and exit points $t_{\pm}$ via 
\begin{align*}
t_{\pm} = \frac{-\xi\cdot(\eta-\xi)\pm\sqrt{(\xi\cdot (\eta-\xi))^2-(\vert\xi\vert^2-1)\vert\eta-\xi\vert^2}}{\vert\eta-\xi\vert}
\end{align*}

Using the parametrizations for $\xi,\eta$ as in \eqref{param} and setting $R:=R_s+R_d$, we obtain 
\begin{align*}
(\P_cu)(\theta,a,b,\zeta) = \frac{2 \sqrt{(R_s \cdot R-b (h \theta+z))^2-\left(a^2+b^2+R^2\right) \left(R_s^2+(-1+h \theta+z) (1+h \theta+z)\right)}}{a^2+b^2+R^2}
\end{align*}


\end{document}