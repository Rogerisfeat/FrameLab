\documentclass[12pt]{article}
\usepackage{framelab}
\usepackage{fullpage}
\usepackage{multirow}

\begin{document}
\begin{center}
Notes for Framelab
\end{center}
\section{Exact Formulae}
Recall the definition of the divergent beam X-ray transform: 

\begin{align*}
(\mathcal{D} u)(a,\theta) = \int_0^\infty u(a+t\theta)dt,\quad a\in\R^n,\theta\in S^{n-1}
\end{align*}

$\mathcal{D}u$ can be extended as a homogenous function to $\R^n\times \R^n$ via 

\begin{align*}
(\mathcal{P}u)(\xi,\eta) = \int_0^\infty u\left(\xi+t\frac{\eta-\xi}{\vert \eta-\xi\vert}\right) dt ,\quad \xi,\eta\in\R^n
\end{align*} In this format, $\xi$ is interpreted as a source position and $\eta$ as a detector position, so that $\frac{\eta-\xi}{\vert \eta-\xi\vert}\in S^{n-1}$ is a unit vector pointing from $\xi$ to $\eta$.  

To parameterize $\mathcal{P}u$ for the cone beam geometry, we use 

\begin{equation}
\begin{aligned}[c]
\xi_1 &= \xi_1(\theta,\alpha,\beta,\zeta) = R_s\cos\theta \\
\xi_2 &= \xi_2(\theta,\alpha,\beta,\zeta) = R_s\sin\theta \\
\xi_3 &= \xi_3(\theta,\alpha,\beta,\zeta) = h\theta + \zeta
\end{aligned}\qquad \qquad
\begin{aligned}[c]
\eta_1 &= \eta_1(\theta,\alpha,\beta,\zeta) = -R_d\cos\theta-\alpha\sin\theta \\
\eta_2 &= \eta_2(\theta,\alpha,\beta,\zeta) = -R_d\sin\theta+\alpha\cos\theta \\
\eta_3 &= \eta_3(\theta,\alpha,\beta,\zeta) = h\theta + \beta + \zeta 
\end{aligned} \label{param}
\end{equation} $R_s,R_d$ are the source-to-isocenter and isocenter-to-detector distances, respectively.  $(\alpha,\beta)$ are the coordinates on the detector plane.  $\theta$ is the gantry rotation angle, and $\zeta$ is a vertical offset.  $h=P/2\pi$ where $P$ is the helix pitch.  We thus obtain the cone beam transform: 
\begin{align*}
(\P_cu)(\theta,\alpha,\beta,\zeta) = \int_0^\infty u\left(\xi+t\frac{\eta-\xi}{\vert \eta-\xi\vert}\right)d\xi
\end{align*}where $\xi=\xi(\theta,\alpha,\beta,\zeta),\eta=\eta(\theta,\alpha,\beta,\zeta)$ are as described in \eqref{param}.  For code validation purposes, we would like to compute $\P_cu$ for some simple functions $u$ when $\xi\in \Gamma\subset\R^3$ is a smooth curve. 

\paragraph{Characteristic Function of the Unit Ball}
The first example we will use is the characteristic function of the unit ball, $u = \chi_{B}(x)$.  The ray transform of $\chi_B(x)$ is simply the length of intersection of the ray $\overrightarrow{\xi\eta}$ with the unit ball.  To compute this length, we use the fact that on $\partial B$ we have
\begin{align*}
\vert x\vert ^2 = 1
\end{align*}  The equation of a line from $\xi$ to $\eta$ is given by 
\begin{align*}
x(t) = \xi+t\frac{\eta-\xi}{\vert \eta-\xi\vert}
\end{align*}thus to find the intersection points, we find where $\vert x(t)\vert^2 = 1$: 
\begin{align*}
\vert x(t)\vert ^2 = \left\vert \xi+t\frac{\eta-\xi}{\vert \eta-\xi\vert}\right\vert ^2  &= \left(\xi+t\frac{\eta-\xi}{\vert \eta-\xi\vert}\right)\cdot \left(\xi+t\frac{\eta-\xi}{\vert \eta-\xi\vert}\right)\\
& = t^2 + 2t\frac{\xi\cdot(\eta-\xi)}{\vert \eta -\xi\vert} + \vert \xi\vert^2
\end{align*}Thus $\vert x(t)\vert^2 = 1$ if 
\begin{align*}
t^2 + 2t\frac{\xi\cdot(\eta-\xi)}{\vert \eta -\xi\vert} + \vert \xi\vert^2-1=0
\end{align*}

This is a quadratic equation in $t$, so we can find the entry and exit points $t_{\pm}$ via 
\begin{align*}
t_{\pm} = \frac{-\xi\cdot(\eta-\xi)\pm\sqrt{(\xi\cdot (\eta-\xi))^2-(\vert\xi\vert^2-1)\vert\eta-\xi\vert^2}}{\vert\eta-\xi\vert}
\end{align*}

Using the parametrizations for $\xi,\eta$ as in \eqref{param} and setting $R:=R_s+R_d$, we obtain 
\begin{align*}
(\P_cu)(\theta,a,b,\zeta) = 2\sqrt{\frac{(R_s \cdot R-b (h \theta+z))^2-\left(a^2+b^2+R^2\right) \left(R_s^2+(-1+h \theta+z) (1+h \theta+z)\right)}{a^2+b^2+R^2}}
\end{align*}


\section{John's Equation}
\subsection{Finite Difference Matrices via Kronecker Products}
Recall the definition of the Kronecker product of two matrices: 
\begin{align*}
A\otimes B = \left(\begin{array}{c|c|c}
a_{11}B &\ldots & a_{1n}B\\\hline
\vdots & \ddots & \vdots \\\hline 
a_{n1}B & \ldots & a_{nn}B
\end{array}\right) 
\end{align*}
First suppose that we have a function $g:\Omega\rightarrow \C$ with 2 arguments $x,y$. For now we assume $\Omega = [0,1]^2$. We discretize $\Omega$ with a Cartesian grid $\Omega_{N_xN_y}: $
\begin{align*}
\Omega_{N_xN_y} = \{(x_i,y_j) : 1\leq i\leq N_x,\quad 1\leq j\leq N_y\},\quad x_i &= (i-1)\Delta x,\quad y_j = (j-1)\Delta y,\\ 
\Delta x &=\frac{1}{N_x-1},\quad \Delta y = \frac{1}{N_y-1}
\end{align*} Then, we will store an approximation to $g$ as a matrix $G$ in matlab, where 
\begin{align*}
G[i,j] \approx g(x_i,y_j)
\end{align*} \textbf{Note: this is the ndgrid() convention, as opposed to the meshgrid() convention.}



Define a one-dimensional interior (no boundary conditions) centered finite difference matrix using Matlab's sparse diagonal matrix function:
\begin{align*}
D^1_x = \text{spdiags([-ones(Nx,1), zeros(Nx,1), ones(Nx,1)], 0:2, Nx-2, Nx  )}
\end{align*}


\begin{align*}
Rg_{\theta b} - RR_s g_{a\zeta} - Rhg_{b\zeta} + 2ag_b + abg_{bb} + (a^2+R^2)g_{ab} = 0
\end{align*}

Adjoint 

\begin{align*}
Rg_{\theta b} - RR_s g_{a\zeta} - Rhg_{b\zeta} - 2(ag)_b + (abg)_{bb} + R^2 g_{ab} + (a^2 g)_{ab} 
\end{align*}

\section{B-Spline Framelets}
We define the B-splines according to \cite{pcmi} as: 
\begin{align*}
\widehat{B}_m(\xi) = e^{-ij\xi/2}\left(\frac{\sin(\xi/2)}{\xi/2}\right)^m,\quad j = m(\text{mod } 2)
\end{align*}
Note that if $m$ is even, this defines a symmetric/centered B-spline, and if $m$ is odd, this defines a `standard' B-spline.
\begin{figure}[ht]
\begin{center}
\def\arraystretch{1.25}
\begin{tabular}{|c|c|c|c|c|}
\hline 
Type & $m$ & $\psi_0$ & $h_0[k]$ & $h_l[k]$  \\\hline\hline  
\multirow{1}{*}{Haar} & 1 & $\psi_0 = \chi_{[0,1]}$ & $\frac{1}{2}[0,1,1]$ & $\frac{1}{2}[0,-1,1]$\\\hline 
\multirow{2}{*}{Linear} & \multirow{2}{*}{2} & \multirow{2}{*}{$\psi_0 = \max\{1-\vert \cdot\vert,0\}$} & \multirow{2}{*}{$\frac{1}{4}[1,2,1]$} & $[-\frac{\sqrt{2}}{4},0,\frac{\sqrt{2}}{4}]$\\
& & & & $[-\frac{1}{4},\frac{1}{2},-\frac{1}{4}]$\\\hline 
\multirow{5}{*}{Cubic} & \multirow{5}{*}{4} & \multirow{5}{*}{$\psi_0(x) = \left\{\begin{array}{ll}
\frac{1}{6}(x+2)^3 & -2\leq x<-1 \\
-\frac{1}{6}(3x^3+6x^2-4) & -1\leq x<0 \\
\frac{1}{6}(3x^3 -6x^2 + 4) & 0\leq x< 1\\
-\frac{1}{6}(x-2)^3 & 1\leq x < 2 \\
0 & \text{else}
\end{array}\right.$} & \multirow{5}{*}{$\frac{1}{16}\left[1,4,6,4,1\right]$ }& $\frac{1}{8}[-1,-2,0,2,1]$ \\
& & & & $\frac{\sqrt{6}}{16}[-1,0,2,-1]$\\
& & & & $\frac{1}{8}[-1,2,0,-2,1]$\\
& & & & $\frac{1}{16}[-1,4,-6,4,-1]$\\
& & & & \\\hline 
\end{tabular}
\end{center}
\caption{Table of framelet data.  Note that $h_l$ are centered i.e. $[0,1/2,1/2] = \left[h_0[-1],h_0[0],h_0[1]\right]$ }
\end{figure}

\bibliographystyle{plain}
\bibliography{/home/nick/Documents/research/bib.bib}
\end{document}