\documentclass[12pt]{article}
\usepackage{fullpage}
\usepackage{framelab}

\usepackage{verbatim}



\title{FrameLab: Development Guide}
\date{\today}

\begin{document}
\maketitle

\section{Overall Design}
FrameLab 1.0 is designed to have an object-oriented, user friendly scripting interface with compute intensive routines written in compiled languages such as C and CUDA/C.  The current scripting language is Matlab, using MEX as an interface mechanism to pull in compiled libraries.  In the future we plan to implement Python/iPython as an alternative to Matlab, to keep the entire code open-source.

\section{Matlab OOP System}
The primary goal of Framelab is to solve general linear inverse problems of the sort 
\begin{align*}
Au = f_0 + \eta 
\end{align*} FrameLab supports models of the type: 
\begin{align*}
\min R(u) \quad \text{such that}\quad F(u)<\epsilon 
\end{align*}where $R(u)$ is a generic regularization term, typically of the form 
\begin{align*}
R(u) = \|Wu\|_1
\end{align*}


\section{Compute Kernels}
\subsection{Computed Tomography}
From \cite{gaocode}

\paragraph{Compiling MEX Libraries}

\begin{verbatim}
mex -L"/usr/local/cuda/lib64" -lcudart -I"./" Ax_fan_mf.cpp Ax_fan_mf_cpu_siddon.cpp
 Ax_fan_mf_cpu_new.cpp Ax_fan_mf_cpu_new_fb.cpp Ax_fan_mf_gpu_siddon.cu 
 Ax_fan_mf_gpu_new.cu Ax_fan_mf_gpu_new_fb.cu find_area.cpp sort_alpha.cpp
\end{verbatim}

Possible error message about invalid conversion fron int to mxComplexity: change 

\begin{verbatim}
plhs[0]=mxCreateNumericMatrix(nx*ny*nt,1,mxSINGLE_CLASS,0);
\end{verbatim} 

to 

\begin{verbatim}
plhs[0]=mxCreateNumericMatrix(nx*ny*nt,1,mxSINGLE_CLASS,mxREAL);
\end{verbatim}

in any mex interface files

\paragraph{Alternating Direction Method of Multipliers}
Recall that ADMM is designed to solve problems of the sort 
\begin{align}
(x^*,y^*) : = \argmin_{x,y} F(x)+G(y)\quad \st\quad  Ax+By = b \tag{$\mathcal{P}$}
\end{align} The approach is to consider the Augmented Lagrangian: 
\begin{align*}
\L_\rho(x,y,\lambda) : = F(x)+G(y) + \left\langle \lambda, Ax+By-b\right \rangle +\frac{\rho}{2}\|Ax+By-b\|_2^2
\end{align*} We then consider the saddle point problem 
\begin{align}
(x^*,y^*,\lambda^*)_\rho = \argmin_{(x,y)}\argmax_\lambda \L_\rho(x,y,\lambda)\label{saddlept} 
\end{align}  Since we are interested in the saddle point itself and not the value of the functionals, we may complete the square in the definition of $\L_\rho$ to obtain 
\begin{align*}
\eqref{saddlept} = \argmin_{(x,y)}\argmax_\lambda F(x) + G(y) +\frac{\rho}{2}\|Ax+By-(b-\lambda/\rho)\|_2^2
\end{align*}For notational convenience, we define
\begin{align*}
\L_\rho^*(x,y,\lambda): =  F(x) + G(y) +\frac{\rho}{2}\|Ax+By-(b-\lambda/\rho)\|_2^2
\end{align*}

 If we then perform coordinate descent/ascent, we arrive at the 3-step ADMM scheme: 
\begin{align*}
\left\{\begin{array}{ll}
x^{(k+1)} &= \argmin_x \L_\rho^*(x,y^{(k)},\lambda^{(k)}) \\
y^{(k+1)} &= \argmin_y \L_\rho^*(x^{(k+1)},y,\lambda^{(k)})\\
\lambda^{(k+1)} &= \argmax_\lambda \L_{\rho}^*(x^{(k+1)},y^{(k+1)},\lambda)
\end{array}\right. 
\end{align*} The method can be generalized in the particular case that $F(x)+G(y)$ is further separable, e.g. $F(x)+G(y) = F_1(x_1)+F_2(x_2)+\ldots+F_n(x_n)$, with $A\textbf{x}=\textbf{b}$, where $\textbf{x} = (x_1,\ldots,x_n)^T$.  The augmented Lagrangian then takes the form 
\begin{align*}
\L_\rho (\textbf{x},\Lambda) : = \sum F_i(x_i) + \left\langle \Lambda,A\textbf{x}-\textbf{b}\right\rangle +\frac{\rho}{2}\|A\textbf{x}-\textbf{b}\|_2^2 
\end{align*} where $\Lambda = (\lambda_1,\ldots,\lambda_n)^T$. 

In FrameLab, we have implemented an ADMM object designed to solve problems of the type $\mathcal{P}$.
\section{Another Section}


\bibliographystyle{plain}
\bibliography{/home/nick/Documents/research/bib}

\end{document}
