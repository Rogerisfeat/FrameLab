\documentclass[12pt]{article}
\usepackage{fullpage}
\usepackage{/home/nick/Documents/research/research}
\usepackage{verbatim}


\title{FrameLab: Development Guide}
\date{\today}

\begin{document}
\maketitle

\section{Overall Design}
FrameLab 1.0 is designed to have an object-oriented, user friendly scripting interface with compute intensive routines written in compiled languages such as C and CUDA/C.  The current scripting language is Matlab, using MEX as an interface mechanism to pull in compiled libraries.  In the future we plan to implement Python/iPython as an alternative to Matlab, to keep the entire code open-source.

\section{Matlab OOP System}
The primary goal of Framelab is to solve general linear inverse problems of the sort 
\begin{align*}
Au = f_0 + \eta 
\end{align*} FrameLab supports models of the type: 
\begin{align*}
\min R(u) \quad \text{such that}\quad F(u)<\epsilon 
\end{align*}where $R(u)$ is a generic regularization term, typically of the form 
\begin{align*}
R(u) = \|Wu\|_1
\end{align*}


\section{Compute Kernels}
\subsection{Computed Tomography}
From \cite{gaocode}

\paragraph{Compiling MEX Libraries}

\begin{verbatim}
mex -L"/usr/local/cuda/lib64" -lcudart -I"./" Ax_fan_mf.cpp Ax_fan_mf_cpu_siddon.cpp
 Ax_fan_mf_cpu_new.cpp Ax_fan_mf_cpu_new_fb.cpp Ax_fan_mf_gpu_siddon.cu 
 Ax_fan_mf_gpu_new.cu Ax_fan_mf_gpu_new_fb.cu find_area.cpp sort_alpha.cpp
\end{verbatim}

Possible error message about invalid conversion fron int to mxComplexity: change 

\begin{verbatim}
plhs[0]=mxCreateNumericMatrix(nx*ny*nt,1,mxSINGLE_CLASS,0);
\end{verbatim} 

to 

\begin{verbatim}
plhs[0]=mxCreateNumericMatrix(nx*ny*nt,1,mxSINGLE_CLASS,mxREAL);
\end{verbatim}

in any mex interface files

\section{Another Section}


\bibliographystyle{plain}
\bibliography{/home/nick/Documents/research/bib}

\end{document}
